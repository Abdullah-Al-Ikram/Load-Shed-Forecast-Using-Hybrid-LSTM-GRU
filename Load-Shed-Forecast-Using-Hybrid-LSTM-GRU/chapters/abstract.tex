


\newpage
    \begin{center}
    
        \vspace*{0.05cm}
        \Large
        \textbf{Abstract}\\
        \vspace{0.5cm}
    \end{center}
    
% Write you abstract here in one paragraph. Cep mapping is provided in another file. Here only the book format is given. All the chapters are in chaperts folder and images are in images folder. See the codes to find out how to add figures and tables. Table of contents, list of figures and tables will be automatically updated when you increase a chapter or add figures or tables. Double click on the content to redirect to its code part. You can include chapters from the main.tex you can check chapter input commands from line 146.


% \chapter*{Abstract}
% \addcontentsline{toc}{chapter}{Abstract}

The increasing frequency of electricity load shedding in rapidly developing regions, particularly in Dhaka City, has prompted the need for reliable forecasting systems to mitigate its adverse impacts. Traditional methods have struggled to predict both the occurrence and magnitude of load shedding events due to their inability to capture complex, non-linear temporal dependencies. This project aims to develop a Hybrid Deep Learning model, combining Gated Recurrent Unit (GRU) and Long Short-Term Memory (LSTM) architectures, for accurate short-term load shedding forecasting. The system utilizes historical data on electricity demand, generation, and weather conditions from the Power Grid Company of Bangladesh (PGCB) to predict both the likelihood of load shedding events (classification) and their magnitude (regression). 
The proposed model demonstrated superior performance, achieving a classification success rate of 91.08\% and low error metrics (MAE: 7.49 MW, RMSE: 28.95 MW) in regression tasks. A comparison with baseline models such as Pure GRU and Pure LSTM further validated the effectiveness of the Hybrid LSTM-GRU model in handling both short-term and long-term dependencies. 
This research provides valuable insights into the application of advanced deep learning techniques in energy grid management and contributes to the development of more efficient and reliable load-shedding prediction systems, with potential applications in improving power grid stability and reducing societal disruption in developing countries. Future work will focus on integrating real-time data and exploring more advanced machine learning architectures to enhance forecasting accuracy and scalability.
\vspace{0.5cm}