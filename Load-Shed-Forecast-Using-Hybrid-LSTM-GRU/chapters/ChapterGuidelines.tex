\color{red}\textbf{Note:}
This chapter is intended solely as a reference for report preparation. It provides formatting and structural guidelines to assist in writing the project report or thesis.

\textbf{This chapter must be removed} from the final version of the formal report submitted for evaluation.
\color{black}
\section*{Chapter Introduction}
Each chapter should begin with a brief introductory paragraph that provides context and sets the stage for the discussion that follows. Typically, this introduction should consist of one to two sentences outlining the purpose, scope, and structure of the chapter.
This helps the reader understand what the chapter covers and how it relates to the overall objectives of the report.

\textbf{Note}: All the chapters are in chaperts folder.

\section*{Chapter Body}
The main body of each chapter should be organized into logical sections and subsections. Each section should address a specific topic or objective, maintaining a coherent flow from general concepts to detailed analysis or results.
Use clear headings and subheadings (e.g., 3.1 System Overview, 3.1.1 Architecture Components) to enhance readability and structure.

\textbf{Writing Tips}
\begin{itemize}
\item Maintain a consistent academic tone and tense throughout.
\item Ensure that every paragraph has a clear main idea supported by evidence, analysis, or explanation.
\item Use numbering consistently for sections, figures, tables, and equations.

\item When discussing methods, results, or analyses, link them back to the objectives outlined earlier in the chapter or report.
\end{itemize}
\section*{Figures and Tables}
Figures and tables are essential for presenting data, system models, architectures, and analytical results effectively. Upload the images in
images folder.

\textbf{Formatting and Numbering}
\begin{itemize}
    \item Number all figures and tables sequentially \textbf{by chapter} (e.g., Figure 3.1, Table 4.2).

\item Place the figure or table close to the relevant text where it is first mentioned.

\item Provide a clear and descriptive \textbf{caption below figures} and\textbf{ above tables}.

\item All figures and tables \textbf{should be referenced} within the text (e.g., “As shown in Figure 3.2, the system architecture comprises three layers…”).

\item Maintain consistent formatting for fonts, line thickness, and labeling in all diagrams.

\item Ensure that figures and tables are self-explanatory and readable, even without referencing the main text.
\end{itemize}


\section*{Sample Table And Figure}

\begin{table}[H]
\centering
\caption{System Components and Technologies Used}
\label{tab:system_components}
\renewcommand{\arraystretch}{1.3}
\begin{tabular}{|p{0.25\textwidth}|p{0.25\textwidth}|p{0.35\textwidth}|}
\hline
\rowcolor{gray!30}
\textbf{Module / Component} & \textbf{Technology Used} & \textbf{Description / Functionality} \\
\hline
User Interface (Frontend) & HTML, CSS, JavaScript, Bootstrap & Provides an interactive web interface for user input and visualization of results. \\
\hline
Backend Server & Python (Flask / Django) & Handles data processing, authentication, and API integration. \\
\hline
Database & MySQL / SQLite & Stores user data, logs, and configuration settings. \\
\hline
Machine Learning Module & TensorFlow / Scikit-learn & Performs model training, prediction, and evaluation. \\
\hline
IoT Sensor Node (if applicable) & NodeMCU, DHT22 Sensor & Collects temperature and humidity data for real-time monitoring. \\
\hline
Cloud / Deployment & AWS / Firebase / Heroku & Enables remote access, storage, and deployment of the application. \\
\hline
\end{tabular}

\end{table}


\textbf{Call of Table: }As shown in Table~\ref{tab:system_components}, the system integrates both web and ML components.

\textbf{Call of Figure:} Figure \ref{fig:Ml_Workflow} illustrates the complete workflow of a machine learning pipeline, comprising five sequential stages—Data Collection, Data Preprocessing, Model Training, Model Evaluation, and Model Deployment.
\vspace{2cm}

\begin{figure}
    \centering
    \includegraphics[width=0.9\linewidth]{images/ML_Workflow.png}
    \caption{Workflow of the machine learning model development process}
    \label{fig:Ml_Workflow}
\end{figure}

\section*{References and Citation Guidelines}

All references in this report must follow the \textbf{APA or IEEE} referencing style. Students are required to cite (such as \cite{ronneberger2015unet} or (Ronneberger et al., 2015)) to all external sources, such as books, journal papers, conference articles, technical reports, and online resources, both within the text and in the reference list.


\section*{Chapter Conclusion}
Each chapter should conclude with a concise summary that reiterates the key concepts, findings, or outcomes presented.
This summary should also link the current chapter to the next one, providing a smooth transition and maintaining logical continuity throughout the document.

\section*{Consistency and Presentation}
To ensure a professional and uniform appearance:
\begin{itemize}
    \item Use consistent formatting for headings, captions, and references throughout the document.

    \item Figures, tables, and appendices should follow the same numbering and referencing conventions.

    \item Maintain uniform margins, font size, and spacing as per the report or thesis formatting guidelines.

    \item Ensure that all acronyms, abbreviations, and symbols are defined at first use.
\end{itemize}